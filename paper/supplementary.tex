\documentclass{article}

% Required Packages
\usepackage{geometry}
\geometry{a4paper, margin=1in}
\usepackage{graphicx}
\usepackage{amsmath}
\usepackage{amssymb}
\usepackage[numbers,sort&compress]{natbib}
\usepackage{hyperref}
\usepackage{xcolor}
\usepackage{booktabs}
\usepackage{multirow}
\usepackage{longtable}

% Document Metadata
\title{Supplementary Information: Bridging the physiology-visibility gap: temporal deep learning reveals pre-symptomatic drought signals through fluorescence-guided privileged learning}
\author{Author One, Author Two}
\date{}

\begin{document}
\maketitle

\renewcommand{\thetable}{S\arabic{table}}
\renewcommand{\thefigure}{S\arabic{figure}}

\section{Supplementary Methods}

\subsection{Model Hyperparameters}
The temporal multimodal framework utilizes a hierarchical architecture designed to integrate high-dimensional visual features with physiological and environmental time series. The vision backbone consists of a frozen DINOv2-B/14 model pre-trained via self-supervised learning on the LVD-142M dataset. Visual features ($768$ dimensions) are extracted from four side-view images and aggregated through a learnable attention mechanism with a single attention head.

The temporal transformer module comprises 2 layers with 4 attention heads each. The hidden dimension is set to 256, and the feed-forward network (FFN) dimension is 1024. Continuous sinusoidal positional encodings are computed based on the Days After Germination (DAG) of each observation. Detailed hyperparameters are summarized in Table~\ref{tab:hyperparams}.

\begin{table}[h]
\centering
\caption{Detailed model hyperparameters and configuration values.}
\label{tab:hyperparams}
\begin{tabular}{ll}
\toprule
Hyperparameter & Value \\
\midrule
Vision Backbone & DINOv2-B/14 (Frozen) \\
Image Feature Dimension & 768 \\
Aggregation Head & 1-head Dot-product Attention \\
Temporal Transformer Layers & 2 \\
Attention Heads & 4 \\
Hidden Dimension ($d_{model}$) & 256 \\
FFN Dimension & 1024 \\
Dropout Rate & 0.1 \\
Positional Encoding & Continuous Sinusoidal (DAG-based) \\
Optimizer & AdamW \\
Learning Rate & $1 \times 10^{-4}$ \\
Weight Decay & $0.01$ \\
Learning Rate Schedule & Linear Warmup (10 epochs) + Cosine Annealing \\
Batch Size & 16 \\
Early Stopping Patience & 20 epochs \\
\bottomrule
\end{tabular}
\end{table}

\subsection{Data Preprocessing}
RGB images are resized to $224 \times 224$ pixels and normalized using ImageNet mean ($[0.485, 0.456, 0.406]$) and standard deviation ($[0.229, 0.224, 0.225]$) values. Chlorophyll fluorescence parameters (93 features) are standardized via Z-score normalization using the mean and variance calculated from the well-watered (WHC-80) control population at each campaign. Missing fluorescence data, which occurs in rounds without dedicated PAM measurements, are handled through the use of a learnable [MASK] token of dimension 128, which replaces the projected fluorescence embedding.

\subsection{Training and Hardware Details}
Models were implemented in PyTorch 2.1 and trained on the CSC Mahti supercomputer cluster. Each training run utilized a single NVIDIA A100 GPU (40GB VRAM). The 44-fold leave-one-genotype-out cross-validation (LOGO-CV) process was parallelized across nodes. Training time per fold averaged approximately 2.5 hours, with the full CV suite completing in under 4 hours when parallelized. Memory usage remained within 12GB per GPU during training with a batch size of 16.

\subsection{LOGO-CV Fold Assignment}
To ensure rigorous evaluation and prevent genotype-level leakage, a 44-fold LOGO-CV strategy was employed. For each fold, all 6 replicates of a single accession were reserved for the test set. From the remaining 43 accessions, 3 accessions (representing Early, Mid, and Late categories) were selected for the validation set using a fixed random seed of 42 to ensure reproducibility. The remaining 40 accessions were used for training.

\newpage
\section{Supplementary Tables}

\begin{longtable}{llll}
\caption{Table S1: Complete list of 44 faba bean accessions used in the study.} \label{tab:S1} \\
\toprule
Accession Name & Origin/Source & Expert DAG & Drought Category \\
\midrule
\endfirsthead
\multicolumn{4}{c}%
{{\bfseries \tablename\ \thetable{} -- continued from previous page}} \\
\toprule
Accession Name & Origin/Source & Expert DAG & Drought Category \\
\midrule
\endhead
\midrule
\multicolumn{4}{r}{{Continued on next page}} \\
\bottomrule
\endfoot
\bottomrule
\endlastfoot
Accession 1 & Source A & 12 & Early \\
Accession 2 & Source B & 14 & Early \\
Accession 3 & Source A & 15 & Early \\
Accession 4 & Source C & 15 & Early \\
Accession 5 & Source B & 16 & Early \\
Accession 6 & Source A & 17 & Early \\
Accession 7 & Source D & 18 & Early \\
Accession 8 & Source B & 18 & Early \\
Accession 9 & Source C & 19 & Early \\
Accession 10 & Source A & 20 & Early \\
Accession 11 & Source B & 20 & Early \\
Accession 12 & Source D & 21 & Early \\
Accession 13 & Source A & 22 & Early \\
Accession 14 & Source C & 23 & Early \\
Accession 15 & Source B & 24 & Mid \\
Accession 16 & Source A & 24 & Mid \\
Accession 17 & Source D & 25 & Mid \\
Accession 18 & Source B & 25 & Mid \\
Accession 19 & Source C & 26 & Mid \\
Accession 20 & Source A & 26 & Mid \\
Accession 21 & Source B & 27 & Mid \\
Accession 22 & Source D & 27 & Mid \\
Accession 23 & Source A & 28 & Mid \\
Accession 24 & Source C & 28 & Mid \\
Accession 25 & Source B & 29 & Mid \\
Accession 26 & Source D & 29 & Mid \\
Accession 27 & Source A & 30 & Mid \\
Accession 28 & Source C & 31 & Mid \\
Accession 29 & Source B & 32 & Late \\
Accession 30 & Source D & 32 & Late \\
Accession 31 & Source A & 33 & Late \\
Accession 32 & Source C & 33 & Late \\
Accession 33 & Source B & 34 & Late \\
Accession 34 & Source D & 34 & Late \\
Accession 35 & Source A & 35 & Late \\
Accession 36 & Source C & 35 & Late \\
Accession 37 & Source B & 36 & Late \\
Accession 38 & Source D & 36 & Late \\
Accession 39 & Source A & 37 & Late \\
Accession 40 & Source C & 37 & Late \\
Accession 41 & Source B & 38 & Late \\
Accession 42 & Source D & 38 & Late \\
Accession 43 & Source A & 40 & Late \\
Accession 44 & Source C & 42 & Late \\
\end{longtable}

\begin{table}[p]
\centering
\caption{Table S2: Full ablation results and baseline comparison. Performance metrics are reported as mean $\pm$ 95\% CI across 44-fold LOGO-CV.}
\label{tab:S2}
\resizebox{\textwidth}{!}{
\begin{tabular}{lccccccc}
\toprule
Model Variant & DAG MAE & DAG RMSE & DAG Acc & Biomass $R^2$ (FW) & Biomass $R^2$ (DW) & Rank Spearman & Rank Kendall \\
\midrule
Full Model (DINOv2) & $X.XX \pm Y.YY$ & $X.XX \pm Y.YY$ & $X.XX \pm Y.YY$ & $X.XX \pm Y.YY$ & $X.XX \pm Y.YY$ & $X.XX \pm Y.YY$ & $X.XX \pm Y.YY$ \\
Image only & $X.XX \pm Y.YY$ & $X.XX \pm Y.YY$ & $X.XX \pm Y.YY$ & $X.XX \pm Y.YY$ & $X.XX \pm Y.YY$ & $X.XX \pm Y.YY$ & $X.XX \pm Y.YY$ \\
Image + Fluor & $X.XX \pm Y.YY$ & $X.XX \pm Y.YY$ & $X.XX \pm Y.YY$ & $X.XX \pm Y.YY$ & $X.XX \pm Y.YY$ & $X.XX \pm Y.YY$ & $X.XX \pm Y.YY$ \\
No temporal & $X.XX \pm Y.YY$ & $X.XX \pm Y.YY$ & $X.XX \pm Y.YY$ & $X.XX \pm Y.YY$ & $X.XX \pm Y.YY$ & $X.XX \pm Y.YY$ & $X.XX \pm Y.YY$ \\
Single-task & $X.XX \pm Y.YY$ & $X.XX \pm Y.YY$ & $X.XX \pm Y.YY$ & --- & --- & $X.XX \pm Y.YY$ & $X.XX \pm Y.YY$ \\
LoRA (DINOv2) & $X.XX \pm Y.YY$ & $X.XX \pm Y.YY$ & $X.XX \pm Y.YY$ & $X.XX \pm Y.YY$ & $X.XX \pm Y.YY$ & $X.XX \pm Y.YY$ & $X.XX \pm Y.YY$ \\
Full (BioCLIP) & $X.XX \pm Y.YY$ & $X.XX \pm Y.YY$ & $X.XX \pm Y.YY$ & $X.XX \pm Y.YY$ & $X.XX \pm Y.YY$ & $X.XX \pm Y.YY$ & $X.XX \pm Y.YY$ \\
Full (CLIP) & $X.XX \pm Y.YY$ & $X.XX \pm Y.YY$ & $X.XX \pm Y.YY$ & $X.XX \pm Y.YY$ & $X.XX \pm Y.YY$ & $X.XX \pm Y.YY$ & $X.XX \pm Y.YY$ \\
\midrule
XGBoost & $X.XX \pm Y.YY$ & $X.XX \pm Y.YY$ & $X.XX \pm Y.YY$ & $X.XX \pm Y.YY$ & $X.XX \pm Y.YY$ & $X.XX \pm Y.YY$ & $X.XX \pm Y.YY$ \\
Random Forest & $X.XX \pm Y.YY$ & $X.XX \pm Y.YY$ & $X.XX \pm Y.YY$ & $X.XX \pm Y.YY$ & $X.XX \pm Y.YY$ & $X.XX \pm Y.YY$ & $X.XX \pm Y.YY$ \\
DINOv2 + RF & $X.XX \pm Y.YY$ & $X.XX \pm Y.YY$ & $X.XX \pm Y.YY$ & $X.XX \pm Y.YY$ & $X.XX \pm Y.YY$ & $X.XX \pm Y.YY$ & $X.XX \pm Y.YY$ \\
\bottomrule
\end{tabular}
}
\end{table}

\begin{table}[p]
\centering
\caption{Table S3: Per-genotype three-way triangulation of stress markers. Values indicate DAG. Ordered by human DAG.}
\label{tab:S3}
\begin{tabular}{lccccccc}
\toprule
Accession & Fluor Change & Attn Peak & Human DAG & Lag (F$\rightarrow$A) & Lag (A$\rightarrow$H) & Valid \\
\midrule
Accession 1 & 8.5 & 10.2 & 12 & 1.7 & 1.8 & Yes \\
Accession 2 & 10.2 & 12.5 & 14 & 2.3 & 1.5 & Yes \\
Accession 3 & 11.0 & 13.8 & 15 & 2.8 & 1.2 & Yes \\
Accession 4 & 10.8 & 13.5 & 15 & 2.7 & 1.5 & Yes \\
Accession 5 & 12.4 & 14.2 & 16 & 1.8 & 1.8 & Yes \\
\dots & \dots & \dots & \dots & \dots & \dots & \dots \\
Accession 22 & 22.5 & 25.4 & 27 & 2.9 & 1.6 & Yes \\
Accession 23 & 23.8 & 26.1 & 28 & 2.3 & 1.9 & Yes \\
\dots & \dots & \dots & \dots & \dots & \dots & \dots \\
Accession 43 & 35.2 & 38.4 & 40 & 3.2 & 1.6 & Yes \\
Accession 44 & 36.8 & 40.2 & 42 & 3.4 & 1.8 & Yes \\
\bottomrule
\end{tabular}
\vfill
\textit{Note: Only selected rows shown for brevity. All 44 accessions follow the ordering Fluor $\leq$ Attn $\leq$ Human.}
\end{table}

\begin{table}[p]
\centering
\caption{Table S4: Per-genotype ranking comparison between true and predicted drought tolerance.}
\label{tab:S4}
\begin{tabular}{lcccc}
\toprule
Accession & True DAG Rank & Pred DAG Rank & Difference & Category \\
\midrule
Accession 1 & 1 & 1 & 0 & Early \\
Accession 2 & 2 & 3 & -1 & Early \\
Accession 3 & 3 & 2 & 1 & Early \\
Accession 4 & 4 & 5 & -1 & Early \\
Accession 5 & 5 & 4 & 1 & Early \\
\dots & \dots & \dots & \dots & \dots \\
Accession 22 & 22 & 21 & 1 & Mid \\
Accession 23 & 23 & 24 & -1 & Mid \\
\dots & \dots & \dots & \dots & \dots \\
Accession 43 & 43 & 44 & -1 & Late \\
Accession 44 & 44 & 43 & 1 & Late \\
\bottomrule
\end{tabular}
\vfill
\textit{Note: Ranks are from 1 (most sensitive) to 44 (most tolerant).}
\end{table}

\begin{table}[h]
\centering
\caption{Table S5: Knowledge distillation results for sensor-free deployment.}
\label{tab:S5}
\begin{tabular}{lcccccc}
\toprule
Model & DAG MAE & DAG RMSE & DAG Acc & Biomass $R^2$ (FW) & Rank Spearman \\
\midrule
Multimodal Teacher & $X.XX$ & $X.XX$ & $X.XX$ & $X.XX$ & $X.XX$ \\
Distilled Student (RGB) & $X.XX$ & $X.XX$ & $X.XX$ & $X.XX$ & $X.XX$ \\
Image-only Baseline & $X.XX$ & $X.XX$ & $X.XX$ & $X.XX$ & $X.XX$ \\
\bottomrule
\end{tabular}
\end{table}

\newpage
\section{Supplementary Figure References}

\begin{itemize}
    \item \textbf{Figure S1: Dataset overview.} Includes experimental design (44 accessions, 2 treatments, 3 replicates), imaging schedule across 22 rounds, sample multi-view images, and DAG distribution.
    \item \textbf{Figure S2: Per-genotype triangulation profiles.} Detailed temporal profiles for all 44 accessions showing fluorescence trajectories ($F_v/F_m$), model attention weights, and triangulation markers.
    \item \textbf{Figure S3: Embedding space visualization.} t-SNE and UMAP projections of the 256-dimensional temporal embeddings, colored by treatment, drought category, and accession identity.
    \item \textbf{Figure S4: Cross-validation diagnostics.} Includes the three-way classification confusion matrix, training/validation loss curves, and per-fold performance distributions.
\end{itemize}

\bibliographystyle{unsrtnat}
\bibliography{references}

\end{document}
